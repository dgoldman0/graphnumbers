\documentclass[11pt]{article}
\usepackage{amsmath, amssymb, amsthm, geometry, thmtools}
\geometry{margin=1in}

\title{\bfseries From Arrows to Operators:\\A Directed Field Completion of Graph Arithmetic}
\author{Daniel Goldman}
\date{April 2025}

\theoremstyle{definition}
\newtheorem{definition}{Definition}[section]
\theoremstyle{plain}
\newtheorem{theorem}[definition]{Theorem}
\newtheorem{proposition}[definition]{Proposition}
\newtheorem{lemma}[definition]{Lemma}
\theoremstyle{remark}
\newtheorem{remark}[definition]{Remark}

\begin{document}
\maketitle

\begin{center}
    \fbox{
        \begin{minipage}{0.92\textwidth}
            \textbf{Draft Note.} This is a first complete draft of the Ordered Graph Reals. It extends the Graph Reals framework to directed graphs by using weighted asymmetric kernels and refined partitions, dispensing with the need for a coordinate pair metric. The document remains a working manuscript and may contain technical errors or incomplete arguments.
        \end{minipage}
    }
\end{center}
\vspace{1em}

\begin{abstract}
This work constructs a topological field of Ordered Graph Reals by embedding directed graphs into a space of weighted, asymmetric step kernels acting on \( L^2([0,1]) \). Beginning from the semiring of finite directed graphs, we adjoin formal differences and fractions to define the Directed Graph Rationals. Each such rational is embedded into a single operator via a refined partition and weight-based step-kernel. We show that the resulting operator norm captures field structure directly—addition, multiplication, and inversion become continuous in the operator norm. Unlike the undirected case, where a pair metric was needed to restore invertibility and injectivity, the directed embedding is sufficient on its own. The resulting field completion subsumes the undirected Graph Reals as a symmetric special case and lays groundwork for a theory of directed graph limits with full algebraic control.
\end{abstract}

\section{Directed Graph Arithmetic}

Let \( \mathcal{N}_G^\to \) denote the isomorphism classes of finite simple directed graphs (no loops, no multiple edges). Define:

\begin{itemize}
  \item Addition: \( [G] + [H] := [G \sqcup H] \)
  \item Multiplication: \( [G] \cdot [H] := [G \Box H] \), the Cartesian product on directed graphs.
\end{itemize}

\begin{definition}[Directed Graph Integers and Rationals]
The Grothendieck group \( \mathcal{Z}_G^\to \) is defined via formal differences:
\[
  [G] - [H] := ([G],[H]) / \sim, \quad \text{where } ([G],[H]) \sim ([G'],[H']) \iff G \sqcup H' \cong G' \sqcup H.
\]
We define the Directed Graph Rationals:
\[
  \mathcal{Q}_G^\to := \left\{ \frac{x}{y} \;\middle|\; x \in \mathcal{Z}_G^\to, y \in \mathcal{Z}_G^\to \setminus \{0\} \right\} / \sim,
\quad
\frac{x}{y} = \frac{x'}{y'} \iff x y' = x' y.
\]
\end{definition}

The ring and field axioms follow exactly as in the undirected case and will not be repeated here.

\section{Weighted Operator Embedding}

\subsection{Refined Step-Kernel Construction}

Let \( G \) be a directed graph on \( n \) vertices. Fix a partition of \([0,1]\) into \( p \ge n \) intervals and a weight function \( w: \{1,\dots,n\}^2 \to \mathbb{R}_{>0} \) such that all weights are distinct. Define the map \( \pi(b) := \lfloor (b-1)n/p \rfloor + 1 \) assigning each block to a vertex.

\begin{definition}[Weighted Step-Kernel]
The associated step-kernel is defined by:
\[
  W_G(x,y) := w(\pi(b_x), \pi(b_y)) \cdot A_{\pi(b_x),\pi(b_y)},
\]
where \( A \) is the adjacency matrix of \( G \) and \( b_x, b_y \) are the partition blocks containing \( x, y \).
\end{definition}

\begin{definition}[Operator Embedding]
For any \( G \), define the bounded linear operator:
\[
  T_G[f](x) := \int_0^1 W_G(x,y)\,f(y)\,dy.
\]
Then extend linearly to all \( x \in \mathcal{Z}_G^\to \) and multiplicatively to all \( \frac{x}{y} \in \mathcal{Q}_G^\to \) via:
\[
  \Phi\left( \frac{x}{y} \right) := T_x T_y^{-1}.
\]
\end{definition}

\section{Metric Structure and Field Completion}

\begin{definition}[Operator Norm Metric]
Define
\[
  d\left( \frac{x}{y}, \frac{x'}{y'} \right) := \left\| T_x T_y^{-1} - T_{x'} T_{y'}^{-1} \right\|_{2 \to 2}.
\]
\end{definition}

\subsection{Well-Definedness and Field Compatibility}

\begin{theorem}
The embedding \( \Phi: \mathcal{Q}_G^\to \to \mathcal{B}(L^2) \) is well-defined, injective, and respects the field operations.
\end{theorem}

\begin{proof}
\textbf{(1) Well-defined:} If \( \tfrac{x}{y} = \tfrac{x'}{y'} \) then \( xy' = x'y \), so
\[
  T_x T_{y'} = T_{x'} T_y \Rightarrow T_x T_y^{-1} = T_{x'} T_{y'}^{-1}.
\]

\textbf{(2) Injectivity:} If \( T_x T_y^{-1} = T_{x'} T_{y'}^{-1} \), then \( T_x T_{y'} = T_{x'} T_y \Rightarrow xy' = x'y \Rightarrow \tfrac{x}{y} = \tfrac{x'}{y'} \).

\textbf{(3) Algebra:} Since \( T \) is a homomorphism, products and sums in the field carry through.

\textbf{(4) Continuity:} Inversion is continuous via the standard operator inequality:
\[
  \| A^{-1} - B^{-1} \| \le \|A^{-1}\| \cdot \|A - B\| \cdot \|B^{-1}\|.
\]
\end{proof}

\subsection{Completion}

\begin{definition}[Ordered Graph Reals]
Define the topological field of Ordered Graph Reals as the completion:
\[
  \widehat{\mathcal{Q}_G^\to} := \overline{\Phi(\mathcal{Q}_G^\to)}
  \subseteq \mathcal{B}(L^2).
\]
\end{definition}

\begin{theorem}
\(\widehat{\mathcal{Q}_G^\to}\) is a complete topological field under the operator norm.
\end{theorem}

\begin{proof}
Closure under addition, multiplication, and inversion follows from norm-continuity of each operation and from the field properties of \( \mathcal{Q}_G^\to \). Since Cauchy sequences converge in \( \mathcal{B}(L^2) \), the space is complete.
\end{proof}

\section{Compatibility with the Undirected Case}

\begin{theorem}
The embedding of undirected Graph Rationals into \( \mathcal{Q}_G^\to \) via symmetric arcs and symmetric weights recovers the original Graph Reals under \( T_G \).
\end{theorem}

\begin{proof}
Let \( G \) be an undirected graph. Construct \( \tilde{G} \) as the symmetric directed graph where \( i \to j \) and \( j \to i \) whenever \( \{i,j\} \in E(G) \). Choose weights such that \( w(i,j) = w(j,i) = 1 \). Then the step-kernel \( W_{\tilde{G}} \) is symmetric and matches the one used in the undirected theory. Therefore, \( T_{\tilde{G}} = T_G \) as before.
\end{proof}

\section{Conclusion}

The Ordered Graph Reals provide a clean, continuous extension of finite directed graphs into an operator-theoretic field. By embedding weighted, direction-aware kernels into \( \mathcal{B}(L^2) \), we are able to dispense with the coordinate-pair embedding used in the undirected case. Every graph fraction is represented by a single invertible operator, and all field operations—including inversion—are continuous under the operator norm. This construction fully generalizes the Graph Reals, supports asymmetric graph structure, and opens the door to a robust analytic theory of directed graph limits.

\end{document}

