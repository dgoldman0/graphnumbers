% ghost_edge_graph_reals.tex
\documentclass[11pt]{article}
\usepackage{amsmath,amssymb,amsthm,geometry}
\geometry{margin=1in}

\title{Graph Reals, the Ghost Edge, and the Unit Graph Stub\\An Algebraic and Analytic Framework}
\author{Daniel Goldman}
\date{April 2025}

% ---------- Theorem environments ----------
\theoremstyle{definition}
\newtheorem{definition}{Definition}[section]
\theoremstyle{plain}
\newtheorem{theorem}[definition]{Theorem}
\newtheorem{proposition}[definition]{Proposition}
\newtheorem{lemma}[definition]{Lemma}
\theoremstyle{remark}
\newtheorem{remark}[definition]{Remark}

% ---------- Symbols ----------
% The unit graph stub (formerly the ``unit cycle'')
\newcommand{\stub}{\mathbb S}
\newcommand{\ghost}{\mathbb E_{\varnothing}}

\begin{document}
\maketitle

\begin{abstract}
We construct the \emph{Graph Reals}~$\mathbb R_G$—the completion of the Grothendieck ring of finite graphs under a combined operator–vertex–edge metric—and focus on two canonical elements: the \emph{unit graph stub} $\stub$ and the \emph{ghost edge} $\ghost$.  We provide full proofs of their convergence, algebraic identities, functional calculus, automorphisms, order properties, and the equivalence of two constructions, and we outline applications to pregeometry and physics.
\end{abstract}

\section{Preliminaries and Definitions}

\begin{definition}[Graph Naturals]
The semiring $\mathcal N_G$ consists of isomorphism classes of finite simple graphs under disjoint union $G\sqcup H$ and Cartesian product $G\times H$.
\end{definition}

\begin{definition}[Graph Integers and Rationals]
The ring $\mathcal Z_G$ is the Grothendieck group completion of $(\mathcal N_G,\sqcup)$, and $\mathcal Q_G$ its field of fractions.
\end{definition}

\subsection{Operator Embedding}
Each graph $G$ on $v$ vertices has adjacency matrix $A_G$.  Partition $[0,1]$ into $v$ equal intervals $I_i$, and define
\[
  W_G(x,y) = A_{ij}\quad(x\in I_i,\;y\in I_j),
\]
with integral operator
\[
  (T_Gf)(x)=\int_0^1 W_G(x,y)f(y)\,dy.
\]
Extend linearly to $\mathcal Z_G$.

\subsection{Pair Metric and Graph Reals}
\begin{definition}[Pair Metric]
For $X,Y\in\mathcal Z_G$ set
\[
  d_{\mathrm{pair}}(X,Y)
  = \max\bigl\{\,\|T_X-T_Y\|_{\mathrm{op}},\;|v(X)-v(Y)|,\;|e(X)-e(Y)|\bigr\},
\]
where $v$ and $e$ denote the vertex‐ and edge‐count homomorphisms.
\end{definition}

\begin{definition}[Graph Reals]
$\mathbb R_G$ is the completion of $(\mathcal Z_G,d_{\mathrm{pair}})$; all ring operations extend continuously.
\end{definition}

\section{The Unit Graph Stub}

\begin{definition}[Stub]
Let $C_n$ be the $n$‑cycle and set $\bar C_n := \tfrac1n C_n\in\mathcal Z_G$.  The \emph{unit graph stub} is
\[
  \stub := \lim_{n\to\infty} \bar C_n = K_1 + \ghost.
\]
Thus $v(\stub)=1$, $e(\stub)=1$, and $T(\stub)=0$.
\end{definition}

\begin{theorem}[Convergence]
$\{\bar C_n\}$ is Cauchy in $d_{\mathrm{pair}}$ and converges to $\stub$.
\end{theorem}
\begin{proof}
$v(\bar C_n)=e(\bar C_n)=1$ for all $n$.  The spectral radius of $A_{C_n}$ is $2$, hence $\|T_{\bar C_n}\|=2/n\to0$.  Therefore $d_{\mathrm{pair}}(\bar C_n,\bar C_m)\le 2/n+2/m\to0$.
\end{proof}

\section{The Ghost Edge}

\begin{definition}
Define $\ghost := \stub - K_1$.  Equivalently, $\bar K_n = 2/[n(n-1)]\,K_n$ converges to $\ghost$.
\end{definition}

\begin{lemma}[Convergence of Complete‐Graph Normalisation]
$\{\bar K_n\}$ is Cauchy in $d_{\mathrm{pair}}$ and converges to an element with $v=0$, $e=1$, $T=0$.
\end{lemma}
\begin{proof}
$v(\bar K_n)=2/(n-1)\to0$, $e(\bar K_n)=1$, and $\|T_{\bar K_n}\|=2/n\to0$.
\end{proof}

\begin{lemma}[Equivalence of Constructions]
The limit of $\bar K_n$ equals $\ghost$.
\end{lemma}
\begin{proof}
Set $D = \lim\bar K_n - (\stub - K_1)$.  Then $v(D)=e(D)=0$ and $T(D)=0$, so $d_{\mathrm{pair}}(D,0)=0$ implies $D=0$.
\end{proof}

\section{Algebraic Identities}
\begin{proposition}
\begin{align*}
  \ghost^{\,2} &= 2\,\ghost,\\[2pt]
  (\stub-K_1)(\stub-3K_1) &= 0,\\[2pt]
  \stub^{-1} &= 2K_1 - \stub = K_1 - \ghost.
\end{align*}
\end{proposition}
\begin{proof}
Compute $v$ and $e$ on products and solve the resulting quadratic in $\ghost$.
\end{proof}

\section{Order‐Theoretic Properties}
\begin{proposition}
In the dominating order,
\[
  K_1 \;\le\; \stub \;\le\; 2\,\stub \;\le\; 3K_1,
\]
so $\ghost=\stub-K_1$ is positive and bounded.
\end{proposition}
\begin{proof}
Use $(\stub-K_1)\ge0$ together with its quadratic relation.
\end{proof}

\section{Decomposition of $\mathbb R_G$}
There is a split short exact sequence
\[
  0\longrightarrow\ker T\longrightarrow \mathbb R_G \xrightarrow{\;T\;}\overline{\Psi(\mathcal Z_G)}\longrightarrow 0.
\]
Every $X\in\mathbb R_G$ decomposes uniquely as
\[
  X \;=\; s(T_X)\; +\; v(X)K_1\; +\;\bigl(e(X)-v(X)\bigr)\,\ghost.
\]

\section{Functional Calculus \& Automorphisms}
Powers obey $\ghost^{\,k} = 2^{\,k-1}\ghost$.  Hence for $f(x)=\sum a_nx^n$,
\[
  f(\ghost) = a_0 + \Bigl(\sum_{n\ge1} a_n 2^{\,n-1}\Bigr)\ghost.
\]
Only two automorphisms of $\ker T$ fix $K_1$.

\section{Physical and Pregeometric Interpretation}
The ghost edge models connectivity‐before‐points, while the stub acts as a dangling half‐edge. Together they furnish order parameters for geometrogenesis, vacuum terms in effective actions, simulation diagnostics, dark‐sector fields, and volume corrections in causal‐set approaches.

\section{Conclusion and Open Issues}
We have supplied complete proofs for the unit graph stub $\stub$ and the ghost edge $\ghost$.  Remaining questions include full metric compatibility for all operations, global continuity of inversion, and detailed analytic applications.

\end{document}

